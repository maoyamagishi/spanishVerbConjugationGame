\documentclass{ltjarticle}
\usepackage{luatexja} % ltjclasses, ltjsclasses を使うときはこの行不要
\usepackage{amsmath,amssymb}
\usepackage{longtable}
\usepackage{listings, xcolor}
\title{PyQt概観}
\author{山岸}

%%%%%%%%%%%%%%  本文  %%%%%%%%%%%%%%%%%%
\begin{document}
\fussy
\maketitle
\begin{abstract}
  PyQtを触りはじめて、FusionのAPIのように各classの従属関係が複雑で、かつ、公式のリファレンス(量が多すぎる)
  ぐらいしか信頼できるドキュメントがないことに気が付いた。\par
  分かったことを順次書いていってドキュメント化することが効率的なやり方だと考えたのでこの文書を作成する。
  各クラスの親子関係、呼び出し、追加の関係をまとめていく。
\end{abstract}


\section{主要モジュール}
まず、いくつかのclassの集まりであり一つ上の概念であるモジュール(module)で主要なものを列挙する。\par
https://doc.qt.io/qt-6/qtmodules.html
によれば主要モジュールは全部で12個あるそうだが、使いそうなのはQt Core, Qt GUI, Qt Widgetsの3つぐらいだろう。
\subsection{Qt Core}
QBasicTimer(読んで字のごとくタイマー機能), 
QEvent(キーボード入力関連),
QFile(ファイルの読み込み、書き込み関連)
などが入っている(重要なものを見つけたら適宜追加する)。
QFileとか、pythonでやるうえでは不要では?cppとかでも他に
やり方がありそうだし、
%https://qiita.com/shirosuke_93/items/d5d068bb15c8e8817c34 <-これコメントアウトしないとエラー吐くんだが…???
まぁ謎。fileExistとかの関数は便利そう。
\subsection{Qt GUI}
Qt coreの拡張(ここでいう拡張は恐らく継承の意味かも)それなのに、
importする時にはPyQt6.QtCoreの下からでは
アクセスできないようになっている、不思議だなぁ。QActionEvent,QColor,QEnterEvent,QImage,QKeyEvent,QText
,iterator(この中にParentFlame?妙だな),QFontなどが入る。
\subsection{Qt Widgets}
本丸である。EventとImage以外大体入ってそうだ。

\section{QGridLayout}
真の名-> PyQt6.QtWigets.QGridLayout()
QHBoxLayoutとかQVBoxLayoutとかもあるが、幅の調整の自由度が低いので、Gridを使いたいところ。
ウィンドウの縮小とかの処理が面倒かなという偏見がある。


\subsection{method}
addLayout()
子レイアウトを埋め込む。多分QHBoxLayoutとかでも使える。\par
addItem()
itemって何ぞ?\par
addWidget()
追加するやつ。QProgressBar()とかQGraphicsView()とかQTextEdit()が追加できる。
\begin{longtable}{c||c}
  QProgressBar() & progress barのインスタンス(widget classに属する)\\
  QTextEdit() & テキスト入力(これもwidget class)
\end{longtable}


\section{QWidget}
QWidgetが、Qtが持つ全てのウィンドウを司るクラスの基になっているらしい。(へ~)
自作したいウィンドウクラスを作成する場合、継承元として指定するぐらいの使い道しか
ないと思われる。紛らわしいけど、真の名をPyQt6.QWidgets.QWidget()。type = <class 'PyQt6.sip.wrappertype'>
\subsection{QWidgetの子孫たち}
こやつらはほとんど同じ扱いができるのでまとめておく。ここでいう同じ扱いとは、\par
・addWidget(instance,arg1,arg2)でGridLayoutに追加できる\\
・hogehoge(発見したら追加)\\
である。
\begin{longtable}{c||c}
  QGraphicsView() & QGraphicScene()を貼り付ける先(e.g. view.setScene(scene))\\
  QProgressBar() & progress barのインスタンス\\
  QTextEdit() & 複数行入れられる入力マス\\
  QLineEdit() & 1行のみの入力マス
\end{longtable}


\section{QGraphicScene}
QGraphicTextItemとか、QGraphicSimpleTextItem()addItemメソッドを使って
貼り付けることができる。また、QtGUI.QPixmapをaddPixmapメソッドを使って貼り付ける
ことができる中間的なオブジェクトである。このオブジェクトをQGraphicsView()に貼り付ける
ことでWidgetを作成でき、さらにそのウィジットをaddWidgetメソッドでレイアウトに
貼り付けることができる。\par
つまり、作成する順番としては(厳密には貼り付ける方向、なのかもしれないが)
(QGraphicSimpleTextItem or QtGUI.QPixmap) -> QGraphicScene
-> QGraphicsView -> Layout() \par
てな感じである。
\lstset{
    basicstyle = {\ttfamily}, % 基本的なフォントスタイル
    frame = {tbrl}, % 枠線の枠線。t: top, b: bottom, r: right, l: left
    breaklines = true, % 長い行の改行
    numbers = left, % 行番号の表示。left, right, none
    showspaces = false, % スペースの表示
    showstringspaces = false, % 文字列中のスペースの表示
    showtabs = false, % タブの表示
    keywordstyle = \color{blue}, % キーワードのスタイル。intやwhileなど
    commentstyle = {\color[HTML]{1AB91A}}, % コメントのスタイル
    identifierstyle = \color{black}, % 識別子のスタイル 関数名や変数名
    stringstyle = \color{brown}, % 文字列のスタイル
    captionpos = t % キャプションの位置 t: 上、b: 下
}
\begin{lstlisting}[caption=sceneSample.py]
  import sys
  from PyQt6.QtWidgets import(QWidget, QGraphicsTextItem,QGraphicsScene,QGraphicsView,QHBoxLayout,QApplication)
  
  class Sample(QWidget):
      def __init__(self) -> None:
          super().__init__()
          self.sampleUI()
          
      def sampleUI(self)->None:
          #create instance of QGraphicTextItem
          txtboxA = QGraphicsTextItem()
          txtboxB = QGraphicsTextItem()
          txtboxA.setPlainText("A")
          txtboxB.setPlainText("B")
          
          #create instance of Scene
          sceneA = QGraphicsScene()
          sceneB = QGraphicsScene()
          
          #create instance of Graphic
          viewA = QGraphicsView()
          viewB = QGraphicsView()
          
          #貼る
          sceneA.addItem(txtboxA)
          sceneB.addItem(txtboxB)
          viewA.setScene(sceneA)
          viewB.setScene(sceneB)
          
          layout = QHBoxLayout()
          layout.addWidget(viewA)
          layout.addWidget(viewB)
          
          self.setLayout(layout)
          self.setGeometry(300,300,350,300)
          self.setWindowTitle('Review')
          self.show()
  
  if __name__ == '__main__':
      app = QApplication(sys.argv)
      smpl = Sample()
      sys.exit(app.exec())
  \end{lstlisting}
\section{tips}
\subsection{How to turn Auto Completion On}
PyQt6をpip installしただけではVSCodeでの予測変換が効かない。これではやってられない(
あるオブジェクトに対してどのmethodやpropatyが使えるかがわかりづらくなる)のでどうにかしよう。\\
"python.autoComplete.extraPaths": $\lbrack$ \\
\qquad \qquad    "C: $\backslash \backslash $ Users $\backslash \backslash$ Appdata...path2PyQt6"\\
$\rbrack$ \\
をsetting.jsonの直下に書き込む
\end{document}