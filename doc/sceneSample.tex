\lstset{
    basicstyle = {\ttfamily}, % 基本的なフォントスタイル
    frame = {tbrl}, % 枠線の枠線。t: top, b: bottom, r: right, l: left
    breaklines = true, % 長い行の改行
    numbers = left, % 行番号の表示。left, right, none
    showspaces = false, % スペースの表示
    showstringspaces = false, % 文字列中のスペースの表示
    showtabs = false, % タブの表示
    keywordstyle = \color{blue}, % キーワードのスタイル。intやwhileなど
    commentstyle = {\color[HTML]{1AB91A}}, % コメントのスタイル
    identifierstyle = \color{black}, % 識別子のスタイル 関数名や変数名
    stringstyle = \color{brown}, % 文字列のスタイル
    captionpos = t % キャプションの位置 t: 上、b: 下
}
\begin{lstlisting}[caption=sceneSample.py]
  import sys
  from PyQt6.QtWidgets import(QWidget, QGraphicsTextItem,QGraphicsScene,QGraphicsView,QHBoxLayout,QApplication)
  
  class Sample(QWidget):
      def __init__(self) -> None:
          super().__init__()
          self.sampleUI()
          
      def sampleUI(self)->None:
          #create instance of QGraphicTextItem
          txtboxA = QGraphicsTextItem()
          txtboxB = QGraphicsTextItem()
          txtboxA.setPlainText("A")
          txtboxB.setPlainText("B")
          
          #create instance of Scene
          sceneA = QGraphicsScene()
          sceneB = QGraphicsScene()
          
          #create instance of Graphic
          viewA = QGraphicsView()
          viewB = QGraphicsView()
          
          #貼る
          sceneA.addItem(txtboxA)
          sceneB.addItem(txtboxB)
          viewA.setScene(sceneA)
          viewB.setScene(sceneB)
          
          layout = QHBoxLayout()
          layout.addWidget(viewA)
          layout.addWidget(viewB)
          
          self.setLayout(layout)
          self.setGeometry(300,300,350,300)
          self.setWindowTitle('Review')
          self.show()
  
  if __name__ == '__main__':
      app = QApplication(sys.argv)
      smpl = Sample()
      sys.exit(app.exec())
  \end{lstlisting}